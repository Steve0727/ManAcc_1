\subsection{Deficiency of Annual Accounting}
\textbf{Financial }accounting is used in the preparation of annual accounts. However, it is of little use in day-to-day management, two disadvantages are present:

\begin{enumerate}
    \item Financial accounts are historical. Problems need to be corrected sooner rather than later. Waiting for financial reports can cripple a business.
    \item They are global. Management may require detailed, or segmental operating results, in addition to the overall position revealed by annual accounts.
\end{enumerate}

Consider Table~\ref{tab:simple} and Table~\ref{tab:detail}. Table~\ref{tab:simple} reveals 20~\% operating profit which may be appear reasonable. However, Table~\ref{tab:detail} indicates that product B is heavily subsidised.

\begin{table}[H]
    \centering
    \begin{tabular}{lr}
    \toprule
                            & \textbf{R} \\
    \midrule
        Sales               & 2 000 \\
        Operating expenses  & 1 600 \\
        Operating profit    &   400 \\
    \bottomrule
    \end{tabular}
    \caption{Simple cash flow statement.}
    \label{tab:simple}
\end{table}

\begin{table}[H]
    \centering
    \begin{tabular}{lrrrr}
    \toprule
    & \multicolumn{1}{c}{Total} & \multicolumn{1}{c}{A} & \multicolumn{1}{c}{B} & \multicolumn{1}{c}{C}\\
    \midrule
        Sales               & 2 000   & 800   & 800   & 500   \\
        Operating expenses  & 1 600   & 400   & 400   & 300   \\
        Operating profit    & 400     & 400   & -200  & 200   \\
    \bottomrule
    \end{tabular}
    \caption{Sales and expenses per product.}
    \label{tab:detail}
\end{table}

Evidently, information can be powerful in determining where costs can be reduced and what segments are profitable, etc.

\subsection{Segmental profit}
Management account may be concerned with operating profit for each department or product. Attempting to detail or costs for a department or job is near impossible, short-cuts are used in practice.

\subsection{Direct Expense}
We must determine expenses related to a particular job. There are typically three types:

\begin{itemize}
    \item \textbf{Direct labour} - labour hours \textbf{directly} employed in the specific job. Priced at the appropriate wage rate.
    \begin{itemize}
        \item All other labour is classified as \textit{indirect labour}, included as an indirect expense.
    \end{itemize}
    \item \textbf{Direct material} - quantities of material directly used in a job. Priced at the appropriate price of the material used.
    \begin{itemize}
        \item Other materials that are used may be classified as \textit{indirect materials}, included as indirect expense.
    \end{itemize}
    \item \textbf{Direct expenses} - other expenses directly related to the given job. Obtained from relevant invoices/receipts etc.
\end{itemize}

$$\sum\text{\bf{Direct expense items}} = \text{\bf{Prime cost for a job}}$$

\subsection{Indirect expense (overheads)}
A large volume of expenditure is not covered by direct expense for a particular job. Generally these expenditure items benefit a number of jobs/departments. The are termed \textbf{indirect expenses} or \textbf{overheads}. The issue now is assigning a fair proportion of overheads to each job/department. A flat percentage does not account for:

\begin{itemize}
    \item Overhead is not incurred uniformly for all jobs. Some sections are more expensive than others etc.
    \item A large portion of overhead is incurred in relation to time. Longer periods relate to larger overhead absorption.
\end{itemize}